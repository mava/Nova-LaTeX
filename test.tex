\documentclass{amsart}
\usepackage[T1]{fontenc}
\usepackage{hyperref}

\title{Proof of the Riemann Hypothesis}
\author{Anonymous}
\date{June 2025}

\newtheorem{theorem}{Theorem}
\newcommand{\myemph}[1]{\textsl{#1}}

\begin{document}

\begin{abstract}
We prove the Riemann Hypothesis.
\end{abstract}

\maketitle
\tableofcontents

\section{Introduction}
\label{intro}

\begin{quote}
``Man findet nun in der That etwa so viel reelle Wurzeln innerhalb dieser Grenzen, und es ist \myemph{sehr wahrscheinlich}, dass alle Wurzeln reell sind.
Hiervon wäre allerdings ein strenger Beweis zu wünschen; ich habe indess die Aufsuchung desselben nach einigen flüchtigen vergeblichen Versuchen vorläufig bei Seite gelassen, da er für den nächsten Zweck meiner Untersuchung entbehrlich schien.'' \cite{Riemann}
\end{quote}

\begin{theorem}
\label{main}
The Riemann Hypothesis is true: all nontrivial zeros of the Riemann zeta function~$\zeta(s)$ have real part equal to~$\frac{1}{2}$.
\end{theorem}

The proof is given in Section~\ref{proof}, and is inspired by Deligne's work~\cite{MR340258} and~\cite{MR601520}.

\section{Proof}
\label{proof}

\begin{enumerate}
\item
\item
    \begin{itemize}
    \item
    \item
    \end{itemize}
\item
\item
\end{enumerate}
Quod Erat Demonstrandum. \qedsymbol

\bibliographystyle{alpha}
\bibliography{test}

\end{document}
